\documentclass[t,usenames,dvipsnames]{beamer}
\usetheme{Copenhagen}
\setbeamertemplate{headline}{} % remove toc from headers
\beamertemplatenavigationsymbolsempty

\usepackage{amsmath, tkz-euclide, tikz, xcolor, pgfplots, array}
\usetkzobj{all}
\pgfplotsset{compat = 1.16}
\usetikzlibrary{arrows.meta, calc, decorations.pathreplacing}
\pgfplotsset{every axis/.append style = {axis lines = middle, axis line style = {<->}}}
\pgfplotsset{every tick label/.append style={font=\scriptsize}}
\everymath{\displaystyle}

\title{Trig Identities}
\author{}
\date{}

\AtBeginSection[]
{
  \begin{frame}
    \frametitle{Objectives}
    \tableofcontents[currentsection]
  \end{frame}
}

\begin{document}

\begin{frame}
    \maketitle
\end{frame}

\section{Derive the Pythagorean identities}

\begin{frame}{Pythagorean Identities}
    These are some of the most important identities in this course. \newline\\
    
    They are based on the Pythagorean Theorem.
\end{frame}

\begin{frame}{Example 1}
Verify $\cos^2\theta + \sin^2\theta = 1$    \\[12pt]
\begin{minipage}{0.3\textwidth}
\begin{tikzpicture}
\tkzDefPoints{0/0/A, 3/0/B, 3/2/C}
\tkzMarkRightAngle(C,B,A)
\tkzDrawPolygon(A,B,C)
\tkzLabelAngle[pos = 0.75](B,A,C){$\theta$}
\tkzLabelSegment[below](A,B){$x$}
\tkzLabelSegment[right](B,C){$y$}
\tkzLabelSegment[above](A,C){$r$}
\end{tikzpicture}
\end{minipage}
\hspace{0.25cm}
\begin{minipage}{0.5\textwidth}
\begin{align*}
\onslide<2->{x^2 + y^2 &= r^2} \\[12pt]
\onslide<3->{\frac{x^2}{r^2} + \frac{y^2}{r^2} &= \frac{r^2}{r^2}} \\[12pt]
\onslide<4->{\left(\frac{x}{r}\right)^2 + \left(\frac{y}{r}\right)^2 &= 1} \\[12pt]
\onslide<5->{(\cos\theta)^2 + (\sin\theta)^2 &= 1} \\[12pt]
\onslide<6->{\cos^2\theta + \sin^2\theta &= 1}
\end{align*}
\end{minipage}
\end{frame}

\begin{frame}{Example 2}
Verify $1 + \tan^2\theta = \sec^2\theta$    \\[12pt]
\begin{minipage}{0.3\textwidth}
\begin{tikzpicture}
\tkzDefPoints{0/0/A, 3/0/B, 3/2/C}
\tkzMarkRightAngle(C,B,A)
\tkzDrawPolygon(A,B,C)
\tkzLabelAngle[pos = 0.75](B,A,C){$\theta$}
\tkzLabelSegment[below](A,B){$x$}
\tkzLabelSegment[right](B,C){$y$}
\tkzLabelSegment[above](A,C){$r$}
\end{tikzpicture}
\end{minipage}
\hspace{0.25cm}
\begin{minipage}{0.5\textwidth}
\begin{align*}
\onslide<2->{x^2 + y^2 &= r^2} \\[12pt]
\onslide<3->{\frac{x^2}{x^2} + \frac{y^2}{x^2} &= \frac{r^2}{x^2}} \\[12pt]
\onslide<4->{1 + \left(\frac{y}{x}\right)^2 &= \left(\frac{r}{x}\right)^2} \\[12pt]
\onslide<5->{1 + (\tan\theta)^2 &= (\sec\theta)^2} \\[12pt]
\onslide<6->{1 + \tan^2\theta &= \sec^2\theta}
\end{align*}
\end{minipage}
\end{frame}

\begin{frame}{Example 3}
Verify $\cot^2\theta + 1  = \csc^2\theta$    \\[12pt]
\begin{minipage}{0.3\textwidth}
\begin{tikzpicture}
\tkzDefPoints{0/0/A, 3/0/B, 3/2/C}
\tkzMarkRightAngle(C,B,A)
\tkzDrawPolygon(A,B,C)
\tkzLabelAngle[pos = 0.75](B,A,C){$\theta$}
\tkzLabelSegment[below](A,B){$x$}
\tkzLabelSegment[right](B,C){$y$}
\tkzLabelSegment[above](A,C){$r$}
\end{tikzpicture}
\end{minipage}
\hspace{0.25cm}
\begin{minipage}{0.5\textwidth}
\begin{align*}
\onslide<2->{x^2 + y^2 &= r^2} \\[12pt]
\onslide<3->{\frac{x^2}{y^2} + \frac{y^2}{y^2} &= \frac{r^2}{y^2}} \\[12pt]
\onslide<4->{\left(\frac{x}{y}\right)^2 + 1 &= \left(\frac{r}{y}\right)^2} \\[12pt]
\onslide<5->{(\cot\theta)^2 + 1 &= (\csc\theta)^2} \\[12pt]
\onslide<6->{\cot^2\theta + 1 &= \csc^2\theta}
\end{align*}
\end{minipage}
\end{frame}

\begin{frame}{Alternate Forms of Pythagorean Identities}
    \[
    \cos^2\theta + \sin^2\theta = 1
    \]
    \pause
    \begin{itemize}
        \item $1 - \sin^2\theta = \cos^2\theta$ \newline\\  \pause
        \begin{itemize}
            \item $(1-\sin\theta)(1+\sin\theta) = \cos^2\theta$ \newline\\  \pause
        \end{itemize}
        \item $1 - \cos^2\theta = \sin^2\theta$ \newline\\  \pause
        \begin{itemize}
            \item $(1+\cos\theta)(1-\cos\theta) = \sin^2\theta$
        \end{itemize}
    \end{itemize}
\end{frame}

\begin{frame}{Alternate Forms of Pythagorean Identities}
    \[
    1 + \tan^2\theta = \sec^2\theta
    \]
    \pause
    \begin{itemize}
        \item $\sec^2\theta - 1 = \tan^2\theta$ \newline\\  \pause
        \begin{itemize}
            \item $(\sec\theta + 1)(\sec\theta - 1)$ \newline\\  \pause
        \end{itemize}
        \item $1 = \sec^2\theta - \tan^2\theta$ \newline\\  \pause
        \begin{itemize}
            \item $1 = (\sec\theta + \tan\theta)(\sec\theta - \tan\theta)$
        \end{itemize}
    \end{itemize}
\end{frame}

\begin{frame}{Alternate Forms of Pythagorean Identities}
    \[
    1 + \cot^2\theta = \csc^2\theta
    \]
    \pause
    \begin{itemize}
        \item $\csc^2\theta - \cot^2\theta = 1$ \newline\\  \pause
        \begin{itemize}
            \item $(\csc\theta + \cot\theta)(\csc\theta - \cot\theta) = 1$ \newline\\  \pause
        \end{itemize}
        \item $\csc^2 \theta - 1 = \cot^2\theta$ \newline\\  \pause
        \begin{itemize}
            \item $(\csc\theta + 1)(\csc\theta - 1) = \cot^2\theta$
        \end{itemize}
    \end{itemize}
\end{frame}


\section{Derive the quotient identities}

\begin{frame}{Quotient Identities}
    The Quotient Identities are
    \[
    \tan \theta = \frac{\sin\theta}{\cos\theta} \quad \text{ and } \quad \cot\theta = \frac{\cos\theta}{\sin\theta}
    \]
\end{frame}

\begin{frame}{Example 4}
Use the triangle to verify $\tan\theta = \frac{\sin\theta}{\cos\theta}$ \\[18pt]
\begin{minipage}{0.3\textwidth}
\begin{tikzpicture}
\tkzDefPoints{0/0/A, 3/0/B, 3/2/C}
\tkzMarkRightAngle(C,B,A)
\tkzDrawPolygon(A,B,C)
\tkzLabelAngle[pos = 0.75](B,A,C){$\theta$}
\tkzLabelSegment[below](A,B){$x$}
\tkzLabelSegment[right](B,C){$y$}
\tkzLabelSegment[above](A,C){$r$}
\end{tikzpicture}
\end{minipage}
\hspace{0.25cm}
\begin{minipage}{0.5\textwidth}
\begin{align*}
\onslide<2->{\frac{\sin\theta}{\cos\theta} &= \dfrac{\left(\dfrac{y}{r}\right)} {\left(\dfrac{x}{r}\right)} } 
\onslide<3->{{\color{blue}\left(\dfrac{r}{r}\right)}} \\[12pt]
\onslide<4->{&=\dfrac{y}{x}} \\[12pt]
\onslide<5->{&=\tan\theta}
\end{align*}
\end{minipage}
\end{frame}


\section{Use the sum and difference identities to verify other identities}

\begin{frame}{Additional Angle Sum and Difference Identities}
\begin{align*}
	\cos(A - B) &= \cos A \cos B + \sin A \sin B \\[6pt]
    \cos(A + B) &= \cos A \cos B - \sin A \sin B    \\[6pt]
    \sin(A - B) &= \sin A \cos B - \sin B \cos A    \\[6pt]
    \sin(A + B) &= \sin A \cos B + \sin B \cos A    \\[6pt]
    \tan(A - B) &= \dfrac{\tan A - \tan B}{1 + \tan A \tan B} \\[8pt]
    \tan(A + B) &= \dfrac{\tan A + \tan B}{1 - \tan A \tan B}   \\[8pt]
    **\text{Note: } \tan(A\pm B) &= \dfrac{\sin(A \pm B)}{\cos(A \pm B)}
\end{align*}    
\end{frame}

\begin{frame}{Example 5}
    Verify $\cos\left(\dfrac{\pi}{2} - \theta\right) = \sin\theta$
    \begin{align*}
        \onslide<2->{\cos\left(\dfrac{\pi}{2} - \theta\right) &= \cos\dfrac{\pi}{2}\cdot \cos\theta + \sin\dfrac{\pi}{2}\cdot \sin\theta} \\[8pt]
        \onslide<3->{&=0\cos\theta + 1\sin\theta} \\[6pt]
        \onslide<4->{&= \sin\theta}
    \end{align*}
\end{frame}

\begin{frame}{Example 6}
    Verify $\cos(-\theta) = \cos\theta$ 
    \begin{align*}
        \onslide<2->{\cos(-\theta) &= \cos(0 - \theta)} \\[6pt]
        \onslide<3->{&= \cos0\cdot \cos\theta + \sin0 \cdot \sin\theta} \\[6pt]
        \onslide<4->{&= 1\cos\theta + 0\sin\theta} \\[6pt]
        \onslide<5->{&= \cos\theta}
    \end{align*}
\end{frame}

\section{Derive double-angle identities}

\begin{frame}{Double-Angle Identities}
The double-angle identities are an extension of the \alert{angle sum identities}. \newline\\

In this case, the angles would be \underline{equal}.    \pause

\begin{align*}
    \onslide<2->{\sin(2A) &= \sin(A + A)} \\[6pt]
    \onslide<3->{&= \sin A \cdot \cos A + \sin A \cdot \cos A} \\[6pt]
    \onslide<4->{&= 2\sin A \cos A}
\end{align*}
\end{frame}

\begin{frame}{Double-Angle Identities}
    \begin{align*}
        \sin(2A) &= 2\sin A \cos A  \\[10pt]
        \cos(2A) &= \cos^2 A - \sin^2 A \\[10pt]
        \tan(2A) &= \frac{2\tan A}{1 - \tan^2 A}
    \end{align*}
\end{frame}

\begin{frame}{Alternate Forms of $\cos (2A)$}
    $\cos (2A)$ has two alternate forms.    \newline\\
    
    Alternate form \# 1:    
    \begin{align*}
        \onslide<2->{\cos(2A) &= {\color{red}\cos^2 A} - \sin^2 A} \\[10pt]
        \onslide<3->{&= {\color{red}1 - \sin^2 A} - \sin^2 A} \\[10pt]
        \onslide<4->{&= 1 - 2\sin^2 A}
    \end{align*}
\end{frame}

\begin{frame}{Alternate Forms of $\cos (2A)$}
    Alternate form \# 2:
    \begin{align*}
        \onslide<2->{\cos(2A) &= \cos^2 A - {\color{red}\sin^2 A}} \\[10pt]
        \onslide<3->{&= \cos^2 A - ({\color{red}1-\cos^2 A})} \\[10pt]
        \onslide<4->{&= \cos^2 A - 1 + \cos^2 A} \\[10pt]
        \onslide<5->{&= 2\cos^2 A - 1}
    \end{align*}
\end{frame}

\section{Derive power-reducing identities}

\begin{frame}{Power-Reduction Identities}
    The power-reduction identities can be derived from the alternate equations for $\cos(2A)$.   \newline\\  \pause
    
    For $\cos(2A) = 2\cos^2A - 1$, if we solve for $\cos^2 A$, we get
    \[
    \cos^2 A = \frac{1+\cos(2A)}{2}
    \]
    \pause
    And for $\cos(2A) = 1 - 2\sin^2 A$, solving for $\sin^2 A$ gives us
    \[
    \sin^2 A = \frac{1-\cos(2A)}{2}
    \]
\end{frame}
\end{document}
