\documentclass[t,usenames,dvipsnames]{beamer}
\usetheme{Copenhagen}
\setbeamertemplate{headline}{} % remove toc from headers
\beamertemplatenavigationsymbolsempty

\usepackage{amsmath, xcolor, tikz, pgfplots, array, bm, amssymb}

\pgfplotsset{compat = newest}
\usetikzlibrary{arrows.meta, calc, decorations.pathreplacing}
\pgfplotsset{every axis/.append style = {axis lines = middle}}
\pgfplotsset{every tick label/.append style={font=\scriptsize}}
\everymath{\displaystyle}

\definecolor{ao}{rgb}{0.0, 0.5, 0.0}
\definecolor{americanrose}{rgb}{1.0, 0.01, 0.24}
\newcommand{\cm}{\color{ao}\checkmark}
\newcommand{\xmark}{\color{americanrose}\textbf{\textsf{X}}}
\newcommand{\?}{\stackrel{?}{=}}

\title{Logarithmic Equations}
\author{}
\date{}

\AtBeginSection[]
{
  \begin{frame}
    \frametitle{Objectives}
    \tableofcontents[currentsection]
  \end{frame}
}

\begin{document}

\begin{frame}
    \maketitle
\end{frame}

\section{Solve logarithmic equations}

\begin{frame}{Logarithmic Equations}
A \alert{logarithmic equation} is one that involves logarithmic functions.  \pause

\[\log_2(x) = 3 \]  \pause
\end{frame}

\begin{frame}{General Techniques for Solving Logarithmic Equations}
\begin{itemize}
    \item Isolate the logarithmic function. \newline\\
    \begin{itemize}
        \item<2-> If convenient, express both sides as logs with the same base and equate the arguments of the log functions.\newline\\
        \item<3-> Else, rewrite the log equation as an exponential equation.
    \end{itemize}
\end{itemize}
\end{frame}

\begin{frame}{For Instance}
    \begin{align*}
        \log_2(x) &= 3 \\[10pt]
        \onslide<2->{2^3 &= x} \\[10pt]
        \onslide<3->{x &= 8}
    \end{align*}
\end{frame}

\begin{frame}{Domain Issues}
    \begin{center}
        \Huge{{\color{red}\textbf{***Important***}}}    \newline\\
        
        \Large{The domain of $\log_b(x)$ is $x > 0$}    \newline\\
        
        \Large{Check your answers!!!}
    \end{center}
\end{frame}

\begin{frame}{Example 1}
Solve each. Round your answers to 4 decimal places. \newline\\
(a) \quad $\log_{117}(1-3x) = \log_{117}(x^2-3)$
\begin{align*}
    \onslide<2->{1-3x &= x^2-3 &\text{Equality Prop.}} \\[6pt]
    \onslide<3->{x^2 + 3x - 4 &= 0 &} \\[6pt]
    \onslide<4->{x &= -4,\ 1&}
\end{align*}
\end{frame}

\begin{frame}{Example 1}
    Checking $x = -4$:  
    \onslide<2->{\quad {\cm}} \newline\\
    \onslide<3->{Checking $x = 1$:}
    \onslide<4->{\quad {\xmark}}
\onslide<5->{\[x = -4\]}
\end{frame}

\begin{frame}{Example 1}
(b) \quad $2- \ln(x-3)=1$
\begin{align*}
    \onslide<2->{-\ln(x-3)&=-1 &} \\[8pt]
    \onslide<3->{\ln(x-3)&= 1 &} \\[8pt]
    \onslide<4->{e^1 &= x-3 &\text{Write in expon. form}} \\[8pt]
    \onslide<5->{x &= e + 3 &} \\[8pt]
    \onslide<6->{x &\approx 5.7183 &}
\end{align*}
\end{frame}

\begin{frame}{Example 1}
    Checking $x = e + 3$:
    \onslide<2->{\quad {\cm}} 
    \onslide<3->{\[ x = e + 3 \approx 5.7183\]}
\end{frame}

\begin{frame}{Example 1}
(c) \quad $\log_6(x+4) + \log_6(3-x) = 1$
\begin{align*}
    \onslide<2->{\log_6((x+4)(3-x)) &= 1 &\text{Prod. Property}} \\[6pt]
    \onslide<3->{\log_6(-x^2-x+12) &= 1 &} \\[6pt]
    \onslide<4->{-x^2-x+12 &= 6 &\text{Write in expon. form}} \\[6pt]
    \onslide<5->{-x^2-x+6 &= 0 &} \\[6pt]
    \onslide<6->{x^2 + x - 6 &= 0 &} \\[6pt]
    \onslide<7->{x &= -3, \, 2 &}
\end{align*}
\end{frame}

\begin{frame}{Example 1 \quad $\log_6(x+4) + \log_6(3-x) = 1$}
    Checking $x = -3$:
    \onslide<2->{\quad {\cm}} \newline\\
    \onslide<3->{Checking $x = 2$:}
    \onslide<4->{\quad {\cm}}
    \onslide<5->{\[ x = -3, \, 2 \]}
\end{frame}

\begin{frame}{Example 1}
(d) \quad $\log_7(1-2x) = 1 - \log_7(3-x)$
\begin{align*}
    \onslide<2->{\log_7(1-2x) + \log_7(3-x) &= 1 &} \\[6pt]
    \onslide<3->{\log_7\left((1-2x)(3-x)\right) &= 1 &\text{Prod. Prop.}} \\[6pt]
    \onslide<4->{\log_7\left(2x^2-7x+3\right) &= 1 &} \\[6pt]
    \onslide<5->{2x^2 - 7x + 3 &= 7 &\text{Write in expon. form}} \\[6pt]
    \onslide<6->{2x^2 - 7x - 4 &= 0 &}  \\[6pt]
    \onslide<7->{x &= -\frac{1}{2}, \, 4&}
\end{align*}
\end{frame}

\begin{frame}{Example 1 \quad $\log_7(1-2x) = 1 - \log_7(3-x)$}
    Checking $x = -\tfrac{1}{2}$
    \onslide<2->{\quad {\cm}} \\[10pt]
    \onslide<3->{Checking $x = 4$}
    \onslide<4->{\quad {\xmark}}
    \onslide<5->{\[x = -\frac{1}{2}\]}
\end{frame}

\begin{frame}{Example 1}
(e) \quad $\log_2(x+3) = \log_2(6-x) + 3$
\begin{align*}
    \onslide<2->{\log_2(x+3)-\log_2(6-x) &= 3 &} \\[8pt]
    \onslide<3->{\log_2\left(\frac{x+3}{6-x}\right) &= 3 &\text{Quotient Prop.}} \\[8pt]
    \onslide<4->{\frac{x+3}{6-x} &= 2^3 &\text{Write in expon. form}} \\[8pt]
    \onslide<5->{\frac{x+3}{6-x} &= 8 &2^3 = 8} \\[8pt]
    \onslide<6->{x+3 &= 8(6-x) &\text{Eliminate the fraction}}
\end{align*}
\end{frame}

\begin{frame}{Example 1 \quad $\log_2(x+3) = \log_2(6-x) + 3$}
\begin{align*}
    x+3 &= 8(6-x) \\[8pt]
    \onslide<2->{x+3 &= 48-8x} \\[8pt]
    \onslide<3->{x &= 5} \\
\end{align*}
\onslide<4->{Checking $x = 5$:}
\onslide<5->{\quad {\cm}}
\end{frame}

\begin{frame}{Example 1}
(f) \quad $1+2\log_4(x+1)=2\log_2(x)$
\begin{align*}
    \onslide<2->{\log_4(x+1) &= \frac{\log_2(x+1)}{\log_2(4)}} \\[8pt]
    \onslide<3->{&= \frac{\log_2(x+1)}{2}}
\end{align*}
\begin{align*}
    \onslide<4->{1 + 2\left(\frac{\log_2(x+1)}{2}\right) &= 2\log_2(x)} \\[8pt]
    \onslide<5->{1 + \log_2(x+1) &= \log_2(x^2)}    \\[8pt]
    \onslide<6->{1 &= \log_2(x^2) - \log_2(x+1)}
\end{align*}
\end{frame}

\begin{frame}{Example 1 \quad $1+2\log_4(x+1)=2\log_2(x)$}
\begin{align*}
    1 &= \log_2\left(\frac{x^2}{x+1}\right) \\[8pt]
    \onslide<2->{2^1 &= \frac{x^2}{x+1}} \\[8pt]
    \onslide<3->{2 &= \frac{x^2}{x+1}} \\[8pt]
    \onslide<4->{2x+2 &= x^2} \\[8pt]
    \onslide<5->{x^2 - 2x - 2 &= 0} \\[8pt]
    \onslide<6->{x &= 1 \pm \sqrt{3}}
\end{align*}
\end{frame}

\begin{frame}{Example 1 \quad $1+2\log_4(x+1)=2\log_2(x)$}
\begin{align*}
    x &= 1 + \sqrt{3} & x &= 1-\sqrt{3} \\[8pt]
    \onslide<2->{x &\approx 2.7321 & x &\approx -0.7321} \\[8pt]
    \onslide<3->{&{\cm} & &{\xmark}} 
\end{align*}
\onslide<4->{\[x \approx 2.7321\] }
\end{frame}
\end{document}
