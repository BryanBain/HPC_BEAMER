\documentclass[t]{beamer}
\usetheme{Copenhagen}
\usepackage{amsmath, tikz, bm, pgfplots,cancel}
\tikzset{>=stealth}
\pgfplotsset{compat=newest}
\pgfplotsset{every tick label/.append style={font=\scriptsize}}
\setbeamertemplate{headline}{} % remove toc from headers
\beamertemplatenavigationsymbolsempty
\everymath{\displaystyle}
% \usepackage[utf8]{inputenc}

\title{Derivatives}
\author{}
\date{}

\AtBeginSection[]
{
  \begin{frame}
    \frametitle{Objectives}
    \tableofcontents[currentsection]
  \end{frame}
}

\begin{document}

\begin{frame}{}
    \maketitle
\end{frame}

\section{Find the derivative of a function}

\begin{frame}{Difference Quotient}
    \[ \frac{f(x+h)-f(x)}{h} \]
\end{frame}

\begin{frame}{Derivative Definition}
The \alert{derivative} is the limit of the difference quotient as $h$ approaches 0.    \newline\\ \pause
\[
f'(x) = \lim_{h\to 0} \frac{f(x+h)-f(x)}{h}
\]
\newline\\  \pause
Other Notations: \quad $\frac{dy}{dx} \qquad y' \qquad \dot{x}$
\end{frame}

\begin{frame}{Example 1}
    Find the derivative of each.    \newline\\
(a) \quad $f(x) = 2x^2 - x + 3$ 
    \begin{align*}
        \onslide<2->{f(x+h) &= 2(x+h)^2 - (x+h) + 3} \\[6pt]
        \onslide<3->{&= 2(x^2 + 2hx + h^2) - x - h + 3} \\[6pt]
        \onslide<4->{&= 2x^2 + 4hx + 2h^2 - x - h + 3} \\[6pt]
        \onslide<5->{f(x+h)-f(x) &= 2x^2+4hx+2h^2-x-h+3-(2x^2-x+3)} \\[6pt]
        \onslide<6->{&= 2x^2+4hx+2h^2-x-h+3 - 2x^2 +x -3} \\[6pt]
        \onslide<7->{&= 4hx - h + 2h^2} 
    \end{align*}
    \end{frame}
    
\begin{frame}{Example 1}
    \begin{align*}
        \frac{f(x+h)-f(x)}{h} &= \frac{4hx - h +2h^2}{h} \\[10pt]
        \onslide<2->{&= 4x - 1 + 2h}    \\[10pt]
        \onslide<3->{&\lim_{h\to 0}(4x - 1 + 2h)} \\[10pt]
        \onslide<4->{&= 4x - 1 + 2(0)} \\[10pt]
        \onslide<5->{&= 4x - 1}
    \end{align*}
    \onslide<6->{\[\alert{f'(x) = 4x - 1}\]} 
\end{frame}

\begin{frame}{Example 1}
(b) \quad $f(x) = -3x^2$   
\begin{align*}
    \onslide<2->{f(x+h) &= -3(x+h)^2} \\[6pt]
    \onslide<3->{&= -3(x^2 + 2hx + h^2)} \\[6pt]
    \onslide<4->{&= -3x^2 - 6hx - 3h^2} \\[6pt]
    \onslide<5->{f(x+h)-f(x) &= -3x^2 - 6hx - 3h^2 - (-3x^2)} \\[6pt]
    \onslide<6->{&= -3x^2 - 6hx - 3h^2 + 3x^2} \\[6pt]
    \onslide<7->{&= -6hx - 3h^2} 
\end{align*}
\end{frame}

\begin{frame}{Example 1}
    \begin{align*}
        \frac{f(x+h)-f(x)}{h} &= \frac{-6hx - 3h^2}{h}  \\[10pt]
        \onslide<2->{&= -6x - 3h} \\[10pt]
        \onslide<3->{&\lim_{h\to 0}(-6x-3h)} \\[10pt]
        \onslide<4->{&= -6x - 3(0)} \\[10pt]
        \onslide<5->{&= -6x}
    \end{align*}
    \onslide<6->{\[ \alert{f'(x) = -6x}\]}
\end{frame}

\begin{frame}{Example 1}
(c) \quad $f(x) = 7x + 9$   
\begin{align*}
    \onslide<2->{f(x+h) &= 7(x+h)+9} \\[6pt]
    \onslide<3->{&= 7x + 7h + 9} \\[6pt]
    \onslide<4->{f(x+h)-f(x) &= 7x + 7h + 9 - (7x + 9)} \\[6pt]
    \onslide<5->{&= 7x + 7h + 9 - 7x - 9} \\[6pt]
    \onslide<6->{&= 7h} 
\end{align*}
\end{frame}


\begin{frame}{Example 1}
    \begin{align*}
        \frac{f(x+h)-f(x)}{h} &= \frac{7h}{h}  \\[10pt]
        \onslide<2->{&= 7} \\[10pt]
        \onslide<3->{&\lim_{h\to 0}(7)} \\[10pt]
        \onslide<4->{&= 7} 
    \end{align*}
    \onslide<5->{\[ \alert{f'(x) = 7}\]}
\end{frame}


\begin{frame}{Example 1}
(d) \quad $f(x) = x^2 + 3$   
\begin{align*}
    \onslide<2->{f(x+h) &= (x+h)^2 + 3} \\[6pt]
    \onslide<3->{&= x^2 + 2hx + h^2 + 3} \\[6pt]
    \onslide<4->{f(x+h)-f(x) &= x^2+2hx+h^2+3-(x^2+3)} \\[6pt]
    \onslide<5->{&= x^2+2hx+h^2+3-x^2-3} \\[6pt]
    \onslide<6->{&= 2hx+h^2} 
\end{align*}
\end{frame}

\begin{frame}{Example 1}
    \begin{align*}
        \frac{f(x+h)-f(x)}{h} &= \frac{2hx+h^2}{h}  \\[10pt]
        \onslide<2->{&= 2x+h} \\[10pt]
        \onslide<3->{&\lim_{h\to 0}(2x+h)} \\[10pt]
        \onslide<4->{&= 2x + 0} \\[10pt]
        \onslide<5->{&= 2x} 
    \end{align*}
    \onslide<6->{\[ \alert{f'(x) = 2x}\]}
\end{frame}

\begin{frame}{Aliases}
    Derivatives have may aliases:   \newline\\  \pause
    \begin{itemize}
        \item Instant rate of change.   \newline\\  \pause
        \item Tangent slope to a curve. \newline\\  \pause
        \item Limit as secant line becomes tangent line.    
    \end{itemize}
\end{frame}

\begin{frame}[c]{Fear Not}
    In calculus, you will learn shortcuts so you won't always have to do what we did in these notes to find the derivative.
\end{frame}

\end{document}
